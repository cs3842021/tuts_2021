\documentclass[12pt,  letterpaper,  twoside]{article}
\usepackage[utf8]{inputenc}
\usepackage{xcolor}
\usepackage{hyperref}

\hypersetup{colorlinks=true,linkcolor=blue,urlcolor=blue}

\title{Tutorial 01 CS384 - Identify Meraki Number}
\author{Dr. Mayank Agarwal}
\date{16th Aug 2021}

\begin{document}
	\maketitle  
	
	
	Write a Python program to check if a given input number is a Meraki number. A number is called Meraki number if all adjacent digits in it differ by 1. All \textbf{single digits }are \textbf{always }considered as Meraki number. Some examples of Meraki number are 0,  5,  10,  12,  78,  567,  101,  6787,  21012.  E.g.,  21012,  (2-1=1),  (1-0=1),  (0-1=1 (mod value)),  (1-2=1(mod value))
	
	
	 Your program must contain a user/programmer-defined function \\  meraki\_helper (n) that prints “yes” or “no” if the input number is Meraki or not along the number (e.g, . Yes - 12 is a Meraki number,  OR No,  72 is Not a meraki number). Finally your program should print the count of meraki numbers and non meraki numbers. E.g., the input list contains 12 meraki and 9 non meraki numbers.
	 
	 Deadline: 18th August 2021,  23:59 \\
	 Output filename: tut01.py \\
	 Push to your Github. \\
	 Your code should execute on
	 \href{https://www.onlinegdb.com/online_python_compiler}{Online Python Compiler - online editor} \\
	 Assume only +ve integers. Python 3 is mandatory.
	 
	 input = [12,  14, 56, 78, 98, 54, 678, 134, 789, 0, 7, 5,  123,  45, 76345,  987654321]
	
\end{document}