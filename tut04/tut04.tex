\documentclass[12pt,  letterpaper,  twoside]{article}
\usepackage[utf8]{inputenc}
\usepackage{xcolor}
\usepackage{hyperref}
\usepackage[margin=1.5cm]{geometry}
\hypersetup{colorlinks=true,linkcolor=blue,urlcolor=blue}

\title{Tutorial 04 CS384 -  Academic Subject Record with Excel Output}
\author{Dr. Mayank Agarwal}
\date{Assignment Given: 10th Sep 2021,\\ Deadline 12th Sep 2021,  
23:59\\Submission: GitHub }

\begin{document}
	\maketitle  
	\textbf{Things to be kept in mind}\\
	\begin{enumerate}
%		\item Dont take any inputs from user . 
		\item You \textbf{need } to use CSV library and OpenPyXL.   
				\item You \textcolor{red}{cannot }use \textbf{pandas} 
				library.   
		\item Program will be checked for plagiarism.  
	\end{enumerate}
	
PS: This is an old list of courses. You may find your roll number here, but 
ignore the details. Its  just for your practise. \\

\textbf{Task 1}:\\ You are given a ``regtable\_old.csv" file containing the 
subjects taken by IITP students. You need to make files for individual roll 
numbers with their subject information.\\

\noindent Template of ``regtable\_old.csv" is as follows:\\
\noindent \textbf{rollno}: Roll number of the student \\
	\textbf{register\_sem}: Semester for which registered. \\	
		\textbf{schedule\_sem}: Semester for which registered	. \\
		\textbf{subno}	: Course Code. \\
		\textbf{grade1}: Ignore. \\
			\textbf{date\_of\_entry1}: Ignore. \\
				\textbf{grade2}: Ignore	. \\
	\textbf{date\_of\_entry2}: Ignore. \\
		\textbf{sub\_type}: Core/Elective. \\


If a roll number has taken $ k $ subjects, there would be $ k $ rows against 
the roll number in the input file   ``regtable\_old.csv". Your task to read 
the  
``regtable\_old.csv" file using the csv library and make $ k $ \textbf{xlsx} 
files 
corresponding to $ k $ roll numbers. A sample output for Task 1 is shown in 
1901EE01.xlsx  file. \\


Check the sample\_output folder
\noindent 1901EE01.xlsx\\ 
rollno,register\_sem,subno,sub\_type\\
1901EE01,5,CS384,Open Elective\\
1901EE01,5,EE330,CORE\\
1901EE01,5,EE331,CORE\\
1901EE01,5,EE350,CORE\\
1901EE01,5,EE370,CORE\\
1901EE01,5,EE372,CORE\\
1901EE01,5,EE381,CORE\\

All of the outputs for Task 1 should go into the folder 
``output\_individual\_roll"
	
\textbf{Task 2}:\\
Now you need to make a file for every individual subject listed in the 
``regtable\_old.csv". Read the 	\textbf{subno} column	and for each individual 
subject that have taken those subject. Check the file  sample\_output 
\textbackslash 
CS384.xlsx file and make for all unique subjects.\\

All the outputs relating to Task 2 should go into ``output\_by\_subject'' 
folder. \\

\noindent I placed the sample output folder just for ease of viewing the 
output. Your outputs should go to folders as described in Task 1 and Task 2.
	
\end{document}